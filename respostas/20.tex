Dada um grafo $(G, \varomega)$, $s, t \in V[G]$ e um valor $k > 0$, mostre como usar Dijkstra para determinar o maior preço entre todos os caminhos de $s$ a $t$ de peso total no máximo $k$. Explique sua ideia sucintamente e escreva um pseudo-código para seu algoritmo. Você não  precisa provar que o algoritmo está correto, mas sua explicação deve ser clara o suficiente para eu me convencer disto. A complexidade do seu algoritmo deve ser $O(V+E + f(V+E))$. Note que você não precisa devolver o caminho, apenas o preço dele.

\begin{enumerate}[label={(\alph*)}]
    \item Nesta questão, todos os grafos são e devem ser representados por listas de adjacências. Além, não têm arestas de peso negativo.

    \item Você tem à disposição um algoritmo chamado Dijkstra que dado um grafo orientado ponderado $(G, \varomega)$ e um vértice $s \in V[G]$, devolve um vetor $d[~]$ indexado por $V[G]$ tal que $d[v] = \mathrm{dist}(s, v)$ para todo $v \in V[G]$. Este algoritmo é dado como uma caixa-preta, i.e., você não sabe como ele é implementado internamente.

    \item Suponha que Dijkstra tem complexidade de tempo $O(f(V+E))$ (uma função de $V$ e $E$). A complexidade exata não é importante, mas deve ser respeitada na solução (veja abaixo).
\end{enumerate}

\itemdsep

O algoritmo é baseado em três informações principais:

\begin{enumerate}
    \item Como o preço de um caminho só depende da aresta de maior peso, não é preciso calcular o preço de todos os caminhos. No caso, a solução será apenas o maior peso $\varomega(u, v)$ dentre todas arestas $u v$ que aparecem em um caminho válido (caminhos de $s$ a $t$ com peso total $\leq k$). Isto é:
    \[
        \max_{C\text{ é válido}}\set{\text{preço}(C)}
        = \max_{C\text{ é válido}} \set{\max_{u v \in C}\set{\varomega(u, v)}}
        = \max_{u v\text{ é parte de um cam. válido}} \set{\varomega(u, v)}
    \]

    \item Para decidir se uma aresta $u v$ é parte de um caminho válido, basta checar se \\ $\dist(s, u) + \varomega(u, v) + \dist(v, t) \leq k$. Note que $\dist(s, u) + \varomega(u, v) + \dist(v, t)$ é o peso total do menor caminho de $s$ a $t$ que passa por $u v$. Se esse caminho não for válido, então nenhum outro que passa por $u v$ será e, portanto, $u v$ não deve ser considerada. Caso contrário, temos um caminho válido com $u v$.

    \item $\dist(s, u)$ é dado diretamente com o resultado de Dijkstra. Para $\dist(v, t)$, no entanto, podemos aplicar o algoritmo novamente, mas no grafo transposto $G^\intercal$, de modo que o resultado é $\dist_{G^\intercal}(t, v) = \dist_{G}(v, t)$ para todo $v \in V[G] = V[G^\intercal]$.

    A função peso $\varomega^*$ de $G^\intercal$ associa para cada aresta transposta $v u \in E[G^\intercal]$ o mesmo peso $\varomega$ da aresta equivalente $u v \in E[G]$ do grafo original.
\end{enumerate}

\begin{codebox}
    \Procname{$\proc{Maior-Preço}(G, \varomega, s, t, k)$}

    \li $d_s[~] \Recebe \proc{Dijkstra}(G, \varomega, s)$
    \li \Seja $\varomega^*$ definida por $\varomega^*(u, v) = \varomega(v, u)$
    \li $d_t[~] \Recebe \proc{Dijkstra}(G^\intercal, \varomega^*, t)$
    \li
    \li $\id{preco} \Recebe -\infty$
    \li \Para \Cada vértice $u \in V[G]$ \Faca
        \Do
    \li     \Para \Cada vértice adjacente $v \in \Adj[u]$ \Faca
            \Do
    \li         \Se $d_s[u] + \varomega(u, v) + d_t[v] \leq k$
                \Do
    \li             \Entao $\id{preco} \Recebe \max\{\id{preco}, \varomega(u, v)\}$
                \End
            \End
        \End
    \li \Devolva $\id{preco}$
\end{codebox}

~

O pseudo-código assume que $k$ é finito. Se for possível que $k = \infty$, seria necessário checar se $d_s[u] < \infty$ e $d_t[v] < \infty$ para garantir que o caminho $(s, \ldots, u, v, \ldots, t)$ existe.
